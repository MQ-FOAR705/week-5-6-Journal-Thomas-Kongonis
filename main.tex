\documentclass{article}
\usepackage[utf8]{inputenc}

\title{journal Wk 5-6}
\author{Thomas Kongonis 44618468 }
\date{August 2019}

\begin{document}

\maketitle

\tableofcontents
\vspace{5mm}
\begin{itemize}
\item{Note: All times are in 24hr format.}
\end{itemize}

\section{Miscellaneous}

\date{27/08/2019 12:34}
\begin{itemize}
\item{Intent:Commit documents to Github easier.}


\item{Action:}
\begin{enumerate}
    \item{Reset Github and Overleaf passwords.}
    \item{Link Github to Overleaf.}
\end{enumerate}

\item{Result: Success, although a difficulty I faced was that I needed to reset all of my passwords due to negligence.}
\end{itemize}
\vspace{5mm}

\section{Data Carpentry: The Unix Shell}
\date{27/08/2019}
\subsection{Lesson 2: Navigating Files and Directories}
\begin{itemize}
    \item{\textbf{Intention: complete instructions 'more ls flags'.}}
\begin{itemize}
   
    \item{\textbf{15:35 Action:} Implementing ls -1 code in the shell.}
    \item{\textbf{Result:} No difference appeared in comparison to just the ls code.}
    \item{\textbf{15:50 Action:} Implementing the ls -1 -h code in the Shell}
    \item{\textbf{Result:} No difference appeared in comparison to just the ls code.}
    \item{\textbf{16:00 Troubleshoot}: consulted ls --help. -s is needed alongside the ls -h function. This was implemented and successfully showed file sizes in a 'human readable' fashion.}
    \end{itemize}
    \vspace{3mm}
    


    \item{\textbf{Intention: Complete instruction 'Listing Recursively and By Time.}}
\begin{itemize}
    \item{\textbf{16:20 Action:} implemented ls -R}
    \item{\textbf{Result:} The directories were most certainly shown recursively. I suppose all these scrolling numbers and letters is the result of a messy computer.}
    \item{\textbf{16:25 Action:} Implement ls -t code.}
    \item {\textbf{Result:} This code does what the lesson said it would do. I made a copy of an old folder with some documents into the home directory and this was shown first.}
    \item{\textbf{16:30 Action:} Implement ls -R -t code.}
    \item{\textbf{Result:} This showed the same ridiculously large amount of information with the only exception of showing the last changed files and directories first.}
\end{itemize}
\end{itemize}

\begin{itemize}
\item{\textbf{Intention:} Complete Activity 'Absolute vs Relative Paths.'}
\begin{enumerate}

\item{\textbf{cd .} NO: the dot means the directory you are in.}
\item{\textbf{cd /} NO: '/' is root directory.}
\item{\textbf{cd /home/amanda} NO: the computer wont understand.}
\item{\textbf{cd ../..} :NO this would make an extra command necessary to move back down again.}
\item{\textbf{cd ~} YES: the tilde would send you to the right place.}
\item{\textbf{cd home} NO: Once again the computer wont read Amanda's mind.}
\item{\textbf{cd ~/data/..} YES: would go to the right spot then down and back up again.}
\item{\textbf{cd} YES: same function as the tilde}
\item{\textbf{cd ..} YES: would go back up one.}


\end{enumerate}
\end{itemize}

\begin{itemize}
\item{\textbf{Intention:} Complete activity 'Relative Path Resolution.' }
\item{\textbf{Response:} The correct answer would be number 4. The other three are not correct as the '..' function would send us back to /Users. The trick in the question is that there are two /Backup directories.}

\item{\textbf{Intention:} Complete activity 'ls Reading Comprehension.}
\item{\textbf{Response:} 1 would not as the pwd command doesnt represent the directory. 2 and 3 are correct. For 2, if we are already in the /Users/backup directory, then the ls command would just show what is inside that directory. 3 is also correct because it specifically tells the shell which directory you want to view but without shortcuts.}

\end{itemize}

\subsection{ Lesson 3: Working With Files and Directories}

\begin{itemize}

\item{\textbf{28/08/2019 11:20 Intention:} Create a Directory.}
\begin{itemize}
\item{\textbf{Action:} In Desktop/data-shell implemented mkdir thesis code.}
\item{\textbf{Result:} Upon utilisng ls code, directory thesis/ was created successfully.}

\end{itemize}

\item{\textbf{28/08/2019 11:33 Intention:} Create a Text File.}
\begin{itemize} 

\item{\textbf{Action:} Implemented code 'nano draft.txt' in /thesis.}
\item{\textbf{Result:} Nano text editor opened successfully in the shell. Text was written and saved without any problems.}

\end{itemize}

\item{\textbf{28/08/2019 11:45 Intention:} Complete Activity 'Creating Files a Different Way'.}
\begin{itemize}
\item{\textbf{Action:} Implement 'touch my file.txt' code.}
\item{\textbf{Result:} Empty .txt file was created in the thesis/ directory.}
\item{\textbf{Question. Why might you want to create a file this way?} Creating a file in this way would have a file there for programs that cannot make their own files to use.}

\end{itemize}

\item{\textbf{12:14 Intention:} Complete Activity 'Moving to the Current Folder'.}

\begin{itemize}

\item{\textbf{Response:} Implement code mv ../analyzed.dat ../analyzed/maltose.dat}
\end{itemize}

\item{\textbf{12:29 Intention:} Complete activity 'Renaming Files'}
\begin{itemize}
\item{\textbf{Solution:}  The correct code to change the file name would be number 2. Number 1 copies the file but the old one remains, number and four provide a lotion without a new name. } 
\end{itemize}

\item{\textbf{ 12:45 Intention:} Complete activity 'Moving and Copying'.}

\begin{itemize}
    \item{\textbf{Response:} We start in the data directory. From here, the code creates a new directory called recombine and proteins.dat is moved to this folder. Following this, a copy is made of this file and its put back into the data folder. This leaves us with the ls function showing just recombine/ as we still haven't left that directory. This means the correct answer is number 2.}
\end{itemize}

\item{\textbf{12:55 Intention:} Complete Activity 'Using rm Safely'.}

\begin{itemize}
\item{\textbf{Action:} Implement 'rm -i quotes.txt'.}
\item{\textbf{Result:} The terminal stopped to give me the option of whether i actually wanted top delete the file, therefore i have inferred that the -i function did this.}
\end{itemize}

\item{\textbf{Intention:} Complete Activity 'copy with multiple file names'.}

\begin{itemize}
\item{\textbf{ 13:00 Action:} implement 'mkdir backup/' implement 'cp amino-acids.txt animals.txt backup/'.}
\item{\textbf{Result:} Files were successfully copied to backup/.}
\item{\textbf{Action:} implement code 'cp amino-acids.txt animals.txt morse.txt'.}
\item{\textbf{Result}. Got a message telling me that morse.txt isnt a directory. Therefore, i would probably need to give a directory after the files to copy to.}

\end{itemize}

\item{\textbf{13:05 Intention:} Complete Activity 'List Filenames Matching a Pattern'.}

\begin{itemize}
\item{\textbf{Action:} implement codes 1-4 on the shell.}
\item{\textbf{Result:} Number 3 successfully showed the two files that the exercise required. This is because both documents have a t and an ne.}
\end{itemize}

\item{\textbf{ 13:10 Intention:} Complete activity 'More on Wildcards'.}
\begin{enumerate}
\item{\textbf{Response:} The first code should be 'cp *calibration.txt backup/calibration'. This because the only differing factor is the dates.}
\item{\textbf{Response} This code requires an asterisk after the date 2015-11 because the directory name is all november files.}
\item{\textbf{Response} The final code needs to include '*-23-dataset*'. This is because Sam wants to send bob all the files from the 23rd in this directory.}
\end{enumerate}

\item{\textbf{13:20 Intention:} Complete Activity 'Organizing Directories and Files'.}

\begin{itemize}
\item{\textbf{Response} If Jamie wants to move these files easily, then the correct response would be to implement 'mv *.dat analyzed/'. This is because the only 2 files in the directory with .dat are the ones that need to be moved.} 
\end{itemize}

\item{\textbf{13:25 Intention:} Complete Activity 'Reproduce a Folder Structure'.}

\begin{itemize}
\item{\textbf{Response:} The first two sets of codes would work. This is because in the first one each directory with the same names and structure are individually created. The second example works because 2 of the sub-directories are copied from the original one. The third example will not work because you would need to create the top directories before the sub-directories. The final example would create the files but without them being correctly formatted into the desired sub-directories.} 
\end{itemize}



\end{itemize}

\subsection{Lesson 4: Pipes and Filters}
\date{16:30 1/09/2019}
\begin{itemize}

\item{\textbf{Intention:} Complete Activity 'What Does sort-n Do?'}
\begin{itemize}

\item{\textbf{Response:} -n stands for numerical so it will sort them in that fashion.}

\end{itemize}

\item{\textbf{Intention:} Complete Activity 'What does >> Mean?'}

\begin{itemize}
\item{\textbf{Action:} implement code 'echo hello > testfile01.txt' twice.}

\item{\textbf{Result:} This code created a text file with the word hello in it. The second code caused no change as the orignial file was rewritten.}

\item{\textbf{Action:} Implement Code 'echo hello >> testfile02.txt' twice.}

\item{\textbf{Result:} The word hello was written to the document twice.}

\end{itemize}

\item{\textbf{Intention:} Complete Activity 'Appending Data'}

\begin{itemize}
\item{\textbf{Action:} Implement codes 'head -n 3 animals.txt > animals-subset.txt' and 'tail -n 2 animals.txt >> animals-subset.txt'.}

\item{\textbf{Result:} The activity gave 4 options and asked that we choose the correct one. The correct answer is 3 as the head code will give 3 lines from the top down and write them to the animal-subset file. Secondly, the tail code will give the last 2 lines of data and add it to the already created file due to the >> code.}

\end{itemize}

\item{\textbf{Intention:} Complete Activity 'Piping Commands Together.'}

\begin{itemize}
\item{\textbf{Action:} Implemented the codes given for the activity except for number 1.}

\item{\textbf{Result:} Based upon my knowledge i knew that the first code would be a problem as the output file writing code was used in liu of the pipe. After trying the other three number 4 was the solution as number 2 didnt work and number three didnt show the 2 test files.}

\end{itemize}

\item{\textbf{Intention:} Complete Activity 'Pipe Reading Comprehension'}

\begin{itemize}
\item{\textbf{Action:} Implement code 'cat animals.txt | head -n 5 | tail -n 3 | sort -r > final.txt'.}

\item{\textbf{Result:} The code does 5 things. First it reads the animals file, then it takes the first 5 lines and extracts the bottom 3 from those five lines. Then, it sorts them in reverse order and writes that to a file. So what we are left with in the file is rabbit, deer then raccoon.}

\end{itemize}

\item{\textbf{Intention:} Complete Activity 'Pipe Construction'}

\begin{itemize}
\item{\textbf{Action:} implement code 'cut -d , -f 2 animals.txt'.}

\item{\textbf{Result:} Got the same result as the lesson. To remove all the repeat animals, we would also need to sort these animals and then only accept the unique ones so we need to add a sort and a uniq and separate them by the pipe. Implemented 'cut -d , -f 2 animals.txt | sort | uniq'. The result was correct.}

\end{itemize}

\item{\textbf{Intention:} Complete Activity 'Which Pipe?'}

\begin{itemize}
\item{\textbf{Action:} implement codes 5, 4 and 3 to see which is the correct coe.}

\item{\textbf{Result:} number 4 is the correct one as it gives a list. number 5 just states that there are 5 unique animals but doesnt specify. Code 3 doesnt sort them and number them correct.}

\end{itemize}

\item{\textbf{Intention:} Complete Activity 'Wildcard Expressions'}

\begin{itemize}
\item{\textbf{Action:} Implemented code ls *A.txt | *B.txt.}

\item{\textbf{Result:} Only showed the B files.}

\item{\textbf{Action:} Implemented code ls *A.txt and the in a new line entered ls *B.txt.}

\item{\textbf{Result:} Got the correct info in 2 seperate results.}

\end{itemize}

\item{\textbf{Intention:} Complete Activity 'Removing Unneeded Files'}

\begin{itemize}
\item{\textbf{Response:} Number one would remove one character name files. Number 2 would remove the correct files. Number 3 would do something weird because of the space. Number 4 would delete everything because the last * would represent any file type.}


\end{itemize}

\end{itemize}

\subsection{Lesson 4: Loops}

\date{1/09/2019 17:00}
\begin{itemize}

\item{\textbf{Intention:} Complete Activity 'Variables in Loops'}


\begin{itemize}
\item{\textbf{Action:} implement first code loop.}

\item{\textbf{Result:} This code listed the same file in rows of 6. it was like a normal ls but done six times in code string.}

\item{\textbf{Action:}  Implement second code loop..}

\item{\textbf{result:} This listed the files one by one. So, it would seem that the first code runs an ls function for every .pdb there is in the directory.}

\end{itemize}

\item{\textbf{Intention:} Complete Activity 'Limiting Sets of Files'}

\begin{itemize}
\item{\textbf{Action:} Implemented first code loop.}

\item{\textbf{Result:} Number 4 is the correct choice. Only cubane.pdb was listed.}

\item{\textbf{Action:} Implement second code loop.}

\item{\textbf{Result:} First time showed me an error which i found out was due to not adding an asterisk after c which was fixed. The second attempt showed cubane and octane.}


\end{itemize}

\item{\textbf{Intention:} Complete Activity 'Saving to a File in a Loop- Part One'}

\begin{itemize}
\item{\textbf{Action:} implemented code in activity.}

\item{\textbf{Result:}. I got error messages 3 times and could not figure out why however on my 4th attempt it was successful. After opening the alkanes.pdb file, i saw that the info from the propane file was in the alkanes file. Therefore 1 is the correct answer but because the less than symbol wasn't used twice it overwrote the file each time.}

\end{itemize}

\item{\textbf{Intention:} Complete Activity 'Saving to a File in a Loop- Part 2'}

\begin{itemize}
\item{\textbf{Action:} Implemented code.}

\item{\textbf{Result:}. Got error message input file is output file multiple times. Decided it wasn't worth wasting more time, will ask in class.}

\end{itemize}

\item{\textbf{Intention:} Complete Activity 'Doing a Dry Run'}

\begin{itemize}
\item{\textbf{Action:} Implement Both codes in exercise.}

\item{\textbf{Result:} The first code actually created files which is not what we wanted as we just wanted a preview. This was because the code told the shell to take what the echo function printed and input it into a new file. The second code, with the quotation marks, tells the shell to echo what the code would do if it were run so that is the correct one.}

\end{itemize}

\item{\textbf{Intention:} Complete Activity 'Nested Loops'}

\begin{itemize}
\item{\textbf{Action:} implement nested code.}

\item{\textbf{Result:} What we have here is a nested code or what it should be named 'codeception'. This code, for the 3 names, runs the sub code on these temperatures and spits out a directory for the relevant temps.}

\end{itemize}

\end{itemize}

\subsection{Lesson 6: Shell Scripts}
\date{05/09/2019 17:40}
\begin{itemize}

\item{\textbf{Intention:} Complete Activity 'List Unique Species'}

\begin{itemize}
\item{\textbf{Action:} Implemented code into shell script.}

\item{\textbf{Result:} Just copied original code and didn't get the correct result, doing what the solution required is beyond me at this stage.}

\end{itemize}

\item{\textbf{Intention:} Complete Activity 'Why Record Commands in the History Before Running Them?'}

\begin{itemize}
\item{\textbf{Response:} It would be good to record the code in case there is a problem with it.}


\end{itemize}


\item{\textbf{Intention:} Complete Activity 'Variables in Shell Scripts'}

\begin{itemize}

\item{\textbf{Result:} The response from the shell that i got for running the created shell code was totally different. I got a bunch of fancy equal signs and arrows around the document titles with the word end beneath.}


\end{itemize}


\item{\textbf{Intention:} Complete Activity 'Find the Longest File With a Given Extension'}

\begin{itemize}

\item{\textbf{Action:} Created shell script and ran.}

\item{\textbf{Result:} I really had no idea what i was doing and it didn't work.}

\end{itemize}

\item{\textbf{Intention:} Complete Activity 'Script Reading Comprehension'}

\begin{itemize}

\item{\textbf{Response:} The first script would make a document for any files with . in their name. The second script would add the contents for the first 3 pdbs to a file. The third script would print all the pdbs.}

\end{itemize}

\item{\textbf{Intention:} Complete Activity 'Debugging Scripts'}

\begin{itemize}
\item{\textbf{Response:} The only thing i could see that i don't know what the result would be is the -x. This has to have something to do with it.}


\end{itemize}


\end{itemize}


\subsection{ Lesson 7: Finding Things}

\date{06/09/2019 8:00}
\begin{itemize}

\item{\textbf{Intention:} Complete Activity 'Using grep'}

\begin{itemize}
\item{\textbf{Action:} Implement codes.}

\item{\textbf{Result:} number 3 was correct because the w function looks for just the word.}

\end{itemize}

\item{\textbf{Intention:} Complete Activity 'Tracking a Species'}

\begin{itemize}
\item{\textbf{Action:} Create script.}

\item{\textbf{Result:} Could not get it to work.}

\end{itemize}

\item{\textbf{Intention:} Complete Activity 'Little Women'}

\begin{itemize}
\item{\textbf{Action:} Attempted creating a shell script.}

\item{\textbf{Result:} Miserable failure, couldn't get it to work.}

\end{itemize}


\item{\textbf{Intention:} Complete Activity 'Matching and Subtracting'}

\begin{itemize}
\item{\textbf{Response:} Number 1 is correct because of the quotation marks that the file name is in.}

\end{itemize}

\item{\textbf{Intention:} Complete Activity 'find Pipeline Reading Comprehension'}

\begin{itemize}
\item{\textbf{Response:} This code will run a word count on files found with a dat extension and then sort them.}



\end{itemize}

\item{\textbf{Intention:} Complete activity 'Finding files with different properties'}

\begin{itemize}
\item{\textbf{Response:} The man function does not word on my shell. That being said the find function is necessary although i cant figue out what else i would need.}


\end{itemize}
\end{itemize}

\vspace{5mm}
\section{Elaboration}
\subsection{Elaboration I}
\date{29/08/2019 14:00}
\begin{itemize}
\item{\textbf{Process:} The first elaboration was done based off the idea that i would be creating a tool that would select singular words. I researched textual analysis and consulted with Shaun about what my next steps should be, and decided to take a new direction from what was elaborated in the first task whilst retaining the same theme.}

\end{itemize}

\subsection{Elaboration II}

\date{05/09/2019 13:00}

\begin{itemize}
\item{\textbf{Process:} The second elaboration was undertaken through the process of testing the ideas that came from consulting with Shaun. Whilst a database with an API couldn't be found, what was found was an amount of public domain translations. As these were not in the format i would like i outlined in the elaboration that part of the proof of concept would be converting these documents into a usable format.}


\end{itemize}
\vspace{5mm}

\section{LateX Coding}
\subsection{Journal Coding}
\date{27/8/2019 12:52}
\begin{itemize}
\item{Intent: Format Week 5-6 Journal in Overleaf}
\item{Actions:}
\begin{enumerate}
    \item{Created new project and implemented table of contents and sections. Action was successful}
    \item{implemented vspace code to place distance between sections. Action was successful}
    \item{Begun by placing dates before each entry utilising date code. Action was successful}
    \item{Implemented item code within first section. Action was unsuccessful and error message was shown.}
    \item{Followed error instructions and found solution:}
    \subitem{https://www.overleaf.com/learn/latex/Lists}
    \item{Implemented begin item and end item codes. Action was successful.}
    \item{Implemented enumerate code within an itemized list as a formatting choice. Action was successful.}
    \item{Tried to paste link shown above into line 52 and was given a formatting error that stated that my line was too long. Solution used was implementing subitem code beneath item number 5. Action was successful.}
    \end{enumerate}
\item{Results:}
\begin{enumerate}
    \item{I learnt that it is incorrect practice to utilize a code without specifically knowing its purpose as this will cause difficulty.}
    \item{I learnt that the formatting functions of \LaTeX are very useful if the time is taken to implement them correctly.}
    \item{Formatting was successfully implemented after some medium difficulty. Paying attention to the error messages as advised in class and dealing with them as they appear is necessary.}
\end{enumerate}
\end{itemize}


\end{document}
