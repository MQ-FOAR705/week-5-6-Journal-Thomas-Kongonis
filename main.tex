\documentclass{article}
\usepackage[utf8]{inputenc}

\title{journal Wk5-6}
\author{Thomas Kongonis 44618468 }
\date{August 2019}

\begin{document}

\maketitle

\tableofcontents
\vspace{5mm}
\begin{itemize}
\item{Note: All times are in 24hr format.}
\end{itemize}

\section{Miscellaneous}

\date{27/08/2019 12:34}
\begin{itemize}
\item{Intent:Commit documents to Github easier.}


\item{Action:}
\begin{enumerate}
    \item{Reset Github and Overleaf passwords.}
    \item{Link Github to Overleaf.}
\end{enumerate}

\item{Result: Success, although a difficulty I faced was that I needed to reset all of my passwords due to negligence.}
\end{itemize}
\vspace{5mm}

\section{Data Carpentry: The Unix Shell}
\date{27/08/2019}
\subsection{Lesson 2: Navigating Files and Directories}
\begin{itemize}
    \item{\textbf{Intention: complete instructions 'more ls flags'.}}
\begin{itemize}
   
    \item{\textbf{15:35 Action:} Implementing ls -1 code in the shell.}
    \item{\textbf{Result:} No difference appeared in comparison to just the ls code.}
    \item{\textbf{15:50 Action:} Implementing the ls -1 -h code in the Shell}
    \item{\textbf{Result:} No difference appeared in comparison to just the ls code.}
    \item{\textbf{16:00 Troubleshoot}: consulted ls --help. -s is needed alongside the ls -h function. This was implemented and successfully showed file sizes in a 'human readable' fashion.}
    \end{itemize}
    \vspace{3mm}
    


    \item{\textbf{Intention: Complete instruction 'Listing Recursively and By Time.}}
\begin{itemize}
    \item{\textbf{16:20 Action:} implemented ls -R}
    \item{\textbf{Result:} The directories were most certainly shown recursively. I suppose all these scrolling numbers and letters is the result of a messy computer.}
    \item{\textbf{16:25 Action:} Implement ls -t code.}
    \item {\textbf{Result:} This code does what the lesson said it would do. I made a copy of an old folder with some documents into the home directory and this was shown first.}
    \item{\textbf{16:30 Action:} Implement ls -R -t code.}
    \item{\textbf{Result:} This showed the same ridiculously large amount of information with the only exception of showing the last changed files and directories first.}
\end{itemize}
\end{itemize}

\begin{itemize}
\item{\textbf{Intention:} Complete Activity 'Absolute vs Relative Paths.'}
\begin{enumerate}

\item{\textbf{cd .} NO: the dot means the directory you are in.}
\item{\textbf{cd /} NO: '/' is root directory.}
\item{\textbf{cd /home/amanda} NO: the computer wont understand.}
\item{\textbf{cd ../..} :NO this would make an extra command necessary to move back down again.}
\item{\textbf{cd ~} YES: the tilde would send you to the right place.}
\item{\textbf{cd home} NO: Once again the computer wont read Amanda's mind.}
\item{\textbf{cd ~/data/..} YES: would go to the right spot then down and back up again.}
\item{\textbf{cd} YES: same function as the tilde}
\item{\textbf{cd ..} YES: would go back up one.}


\end{enumerate}
\end{itemize}

\begin{itemize}
\item{\textbf{Intention:} Complete activity 'Relative Path Resolution.' }
\item{\textbf{Response:} The correct answer would be number 4. The other three are not correct as the '..' function would send us back to /Users. The trick in the question is that there are two /Backup directories.}

\item{\textbf{Intention:} Complete activity 'ls Reading Comprehension.}
\item{\textbf{Response:} 1 would not as the pwd command doesnt represent the directory. 2 and 3 are correct. For 2, if we are already in the /Users/backup directory, then the ls command would just show what is inside that directory. 3 is also correct because it specifically tells the shell which directory you want to view but without shortcuts.}




\end{itemize}

\vspace{5mm}
\section{Elaboration}

\vspace{5mm}

\section{LateX Coding}
\subsection{Journal Coding}
\date{27/8/2019 12:52}
\begin{itemize}
\item{Intent: Format Week 5-6 Journal in Overleaf}
\item{Actions:}
\begin{enumerate}
    \item{Created new project and implemented table of contents and sections. Action was successful}
    \item{implemented vspace code to place distance between sections. Action was successful}
    \item{Begun by placing dates before each entry utilising date code. Action was successful}
    \item{Implemented item code within first section. Action was unsuccessful and error message was shown.}
    \item{Followed error instructions and found solution:}
    \subitem{https://www.overleaf.com/learn/latex/Lists}
    \item{Implemented begin item and end item codes. Action was successful.}
    \item{Implemented enumerate code within an itemized list as a formatting choice. Action was successful.}
    \item{Tried to paste link shown above into line 52 and was given a formatting error that stated that my line was too long. Solution used was implementing subitem code beneath item number 5. Action was successful.}
    \end{enumerate}
\item{Results:}
\begin{enumerate}
    \item{I learnt that it is incorrect practice to utilize a code without specifically knowing its purpose as this will cause difficulty.}
    \item{I learnt that the formatting functions of \LaTeX are very useful if the time is taken to implement them correctly.}
    \item{Formatting was successfully implemented after some medium difficulty. Paying attention to the error messages as advised in class and dealing with them as they appear is necessary.}
\end{enumerate}
\end{itemize}


\end{document}
